\documentclass[spanish,a4paper,11pt]{article}

\usepackage{latexsym,amsfonts,amssymb,amstext,amsthm,float,amsmath}
\usepackage[spanish]{babel}
\usepackage[utf8]{inputenc}
\usepackage[dvips]{epsfig}
\usepackage{doc}
\usepackage{graphicx}

\begin{document}
\title{Aproximación del número $\pi$}
\author{Cathaysa Pérez Quintero \\ Práctica \#10}
\date{9 de abril de 2014}

\maketitle

\begin{abstract}
El objetivo es entregar un programa escrito en Python en el que se aproxime el valor de $\pi$. Además de
su informe escrito en \LaTeX{}.
\end{abstract}

\section{Motivación y objetivos}

A lo largo de la historia han sido muchas las formas utilizadas por el
ser humano para calcular aproximaciones cada vez más exactas del número $\pi$.\\
El objetivo de esta práctica de laboratorio es implementar el código \textsf{Python}
que permita a\-pro\-xi\-mar el número $\pi$ con una cierta precisión.\\
$\pi$ se puede calcular mediante integración:

$$\int_{0}^{1} \! \frac{4}{1+x^2}\, dx = 4(atan(1) -atan(0)) = \pi $$

Esta integral\footnote{Cálculo matemático usualmente utilizado} se puede aproximar numéricamente con una fórmula de cuadratura.

\subsection{Regla del punto medio}

Si se utiliza la regla del punto medio se obtiene:

\begin{center}
$ \pi \approx \frac{1}{n} \sum\limits_{i=1}^{n}f(x_i)\,$,
con $f(x) = \frac{4}{(1+x^2)}\,$,
$x_i = \frac{i - \frac{1}{2}}{n}$,
para $i = 1, \dots, n$
\end{center}

\section{Ejercicios propuestos}

Escriba un programa que reciba como entrada el número de subintervalos
con los que se desea abordar la aproximación del número $\pi$.\\

A partir de él se deben calcular y mostrar por la consola:

\begin{enumerate}
  \item
    Los extremos de los subintervalos.
  \item
    El punto $x_i$.
  \item
    El valor de de la función de aproximación de $pi$, $f(x_i)$.
  \item
    El resultado de la aproximación.
  \item
    La constante $pi$ con treinta y cinco decimales.
\end{enumerate}

\subsection{Ejemplo}
Por ejemplo, si se utilizan 4 subintervalos, la salida debería ser: 
\begin{footnotesize}
\begin{verbatim}
Introduzca el número de intervalos (n > 0): 4
Subintervalo: [0 , 0.25] x_i: 0.125 fx_i: 3.93846
Subintervalo: [0.25, 0.5 ] x_i: 0.375 fx_i: 3.50685
Subintervalo: [0.5 , 0.75] x_i: 0.625 fx_i: 2.8764
Subintervalo: [0.75, 1 ] x_i: 0.875 fx_i: 2.26549

El valor aproximado de PI es: 3.14680051839
\end{verbatim}
\end{footnotesize}

En forma de tabla:\\
\begin{table}[!h]
\begin{tabular}{lrc}
Subintervalo & xi & fxi\\
\hline
0 , 0.25 & 0.125 & 3.93846\\
0.25, 0.5  & 0.375 & 3.50685\\
0.5 , 0.75 & 0.625 & 2.8764\\
0.75, 1  & 0.875 & 2.26549
\end{tabular}
\caption{Es una tabla del ejercicio anterior}
\label{Mitabla}
\end{table}

En la tabla \ref{Mitabla} aparece el ejemplo anterior.

\section{Imagen}
\begin{figure}[h]
\includegraphics[scale=0.15]{imagen1.eps}
\caption{Imagen ULL}
\label{Mifigura}
\end{figure}

En la figura \ref{Mifigura} aparece el logo de la Universidad. 

\end{document}